\chapter{College of Engineering Formatting Requirements}
\label{ap:format}

Theses and dissertations in the College of Engineering should follow the requirements outlined below.

\begin{itemize}
 \item {\bfseries BYU Graduate Studies requirements:}  The document must include a title page, abstract, and table of contents. The title page must follow the format prescribed at \url{https://gradstudies.byu.edu/academics/thesis-and-dissertation}.  

\item {\bfseries PDF requirements:} All fonts should be embedded in the PDF. The PDF file should include bookmarks for each chapter and heading that is present in the table of contents section.

\item {\bfseries Page size, margins, and line spacing:} US letter-sized pages with margins of at least 1~inch on all sides of the text column should be used. Line spacing should be single spaced and facilitate readability with no more than six lines per inch. If using notes and figures in the side margin as is done with the {\ttfamily fancy} class option of {\ttfamily byuthesis.cls}, the space between the text column and the side-margin column must be at least 0.125~inches and the space between the side-margin column and the edge of the page must be at least 0.5~inches. 

\item {\bfseries Fonts:} The text font must be a conservative serif-styled font (e.g., Palatino, Times New Roman, Garamond), size 11 or 12 pt. The font style and size must be consistent throughout the text. A 10 pt font size is allowed for figure and text captions. Font sizes for figures should be selected so that the text easily legible on a printed page. A sans-serif font can be used for chapter and section headings if desired.

\item {\bfseries Use of color:} The main text of the document should be black in color. Color can be used judiciously in the document for items such as chapter names and numbers, section names and numbers, and figure and table labels. Colors of text should be limited to black, grey ({\ttfamily 0x666666}), navy blue ({\ttfamily 0x002E5D}), and royal blue ({\ttfamily 0x005CAB}) as specified by university style guides at \url{https://brand.byu.edu/colors}.

\item {\bfseries Page numbering:} Preliminary matter should be numbered with lowercase Roman numerals (i, ii, and so forth) starting with the Table of Content page. Main matter beginning with the first page of the first chapter should be numbered consecutively with Arabic numerals starting with 1. 

\item {\bfseries Chapters, sections, and appendices:} The document should be divided into chapters. Within chapters, section and subsection headings should be set off with titles. Appendices may be included after the list of references.

\item {\bfseries References:} Works cited in the document should be included in a list of references after the last chapter and before appendices. References should be cited in the text using a standard format such as (author, year) or by number (e.g., [1]). The list of references should follow a standard format and must include sufficient information for the work to be located. 

\item {\bfseries Equations:} Displayed equations should be numbered in the format (chapter.number), so that the first equation in Chapter 2 is numbered (2.1) and referenced in the text in the same format. 

\item {\bfseries Figures and tables} should be numbered in the format Figure chapter.number, as in Figure 2.1 or Table 3.2. Tables and figures appearing in appendices should be numbered A.1, A.2, B.1, and so forth. Figures should include a descriptive caption below the figure. Tables should include a descriptive title above the table. 

\item {\bfseries Article-based chapters:} If permitted by the thesis and dissertation formatting requirements of the department, submitted, accepted, or published articles for which the student is a primary author may be inserted as chapters in the thesis or dissertation. The inserted article text must be formatted in the same format as the other chapters of the document with consistent page numbering. Article-based chapters should provide a citation to the article and a brief statement of its publication status at the time of submission of the thesis/dissertation.
\end{itemize}