\chapter{Article-based Chapters}
\label{ch:article_based_chap}

With the recent changes in thesis and dissertation formatting, BYU Graduate Studies now allows students to insert journal or conference articles as chapters into their thesis/dissertation document. If approved by the department, a student in the College of Engineering may insert article-based chapters into their document. To do so, the student must be a primary author. The formatting of an article-based chapter must be fully consistent with the formatting defined in this document with chapter, section, equation, figure, and table numbering integrated accordingly. Article-based chapters must include a complete citation and the following statement: ``I hereby confirm that the use of this article is compliant with all publishing agreements.'' The paragraph below is an example of how this could be done. It should appear immediately following the chapter title.

\vspace{0.1in}
\noindent This chapter is composed from a paper entitled ``Really great research from a BYU engineering student'' published in the journal Awesome Engineering~\cite{StudentRP20}. I hereby confirm that the use of this article is compliant with all publishing agreements.



\section{Some Additional Comments}
The publication of a conference or journal article is a significant milestone for a graduate  student and should be an objective for all students pursuing graduate research in the College of Engineering. We encourage the use of article-based chapters in theses and dissertations provided that it aligns with the goals and objectives of the research. Articles, however, are often constrained in length forcing the exposition to be more concise or narrow in scope than may be desired for the intended audience of the thesis/dissertation. For example, if an objective is to guide the learning of subsequent graduate-student researchers, it may be beneficial to include additional details or a broader discussion that may be more tutorial in nature. Often more results from a wider variety of cases would be included in a dissertation or thesis than would be possible in a journal or conference publication. Your graduate committee will help guide your efforts in these matters.
